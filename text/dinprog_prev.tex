\documentclass[12pt]{article}
\usepackage{a4wide}
\usepackage[utf8]{inputenc}
\usepackage[russian]{babel}
\usepackage[dvips]{graphicx, color}
\usepackage{epstopdf}
\usepackage{amsmath}
\usepackage{amsfonts}
\usepackage{amssymb}
\usepackage{amsthm}

\begin{document}

\thispagestyle{empty}

\begin{center}
\ \vspace{-3cm}

\includegraphics[width=0.5\textwidth]{msu.eps}\\

{\scshape Московский государственный университет имени М.~В.~Ломоносова}\\
Факультет вычислительной математики и кибернетики\\
Кафедра системного анализа

\vfill

{\LARGE Отчёт по практикуму}

\vspace{1cm}

{\Huge\bfseries <<Динамическое программирование и процессы управления>>} \\

\end{center}

\vspace{1cm}

\begin{flushright}
  \large
  \textit{Студент 415 группы}\\
  Е.~Н.~Грачев

  \vspace{5mm}

  \textit{Руководитель практикума}\\
  Ю.~Ю.~Минаева
\end{flushright}

\vfill

\begin{center}
Москва, 2014
\end{center}

\pagebreak
\tableofcontents
\pagebreak

\newtheorem{definition} {Определение}
\newtheorem{option} {Свойство}
\newtheorem{theorem} {Теорема}


\section{Постановка задачи}
    		Дана линейная управляемая система:
    		$$
    			\begin{cases}
    				\dot{x}(t) = A(t) x(t) + B(t) u(t), t \in [t_0, t_1], \\
    				x(t_0) \in \mathcal{E}(x_0, X_0), \\
    				u(t) \in \mathcal{E} (p(t), P(t)), \\
    				u(t) \in \mathbb{R}^m, x(t) \in \mathbb{R}^n. 
    			\end{cases}
    		$$
		Построить внутренние эллипсоидальные оценки множества и трубки достижимости системы при наличии фазового ограничения $x(t) \in \mathcal{Y}(t)$. Фазовое ограниченеие задано в виде выпуклого компакта $\mathcal{Y}(T) = \{x = (x_1, \ldots, x_n)\ |\ |x_i(t)| \leqslant y_i(t), \ i = 1, \ldots, n \}$
    		В качестве примера рассмотреть следующую колебательную систему:
       	Двойной математический маятник состоит из двух невесомых стержней длины $l_1$ и $l_2$ и двух грузов массой $m$. Маятник совершает малые колебания в вертикальной плоскости. К нижнему шарику приложено управляющее ускорение $u$. 
		Задано фазовое ограничение: угол отклонения верхнего груза от вертикальной оси $\varphi_1(t)$ не может превышать по модулю некоторой заданной величины $|\varphi_1(t)| \leqslant y_1$, $y_1 > 0$.

		\textbf{Требования к программе:}
			Для выполнения задания разрешается использовать функции Ellipsoidal Toolbox для задания эллипсоидов, поиска пересечения эллипсоида и гиперплоскости (и множества $\mathcal{Y}(t)$). Не разрешается использоваться функции Ellipsoidal Toolbox для построения проекции эллипсоида на плоскость, построения внутренней или внешней оценки геометрической суммы и разности эллипсоидов, для пересечения и объединения эллипсоидов.
			Изобразить полученные эллипсоидальные оценки следующими способами:
			\begin{enumerate}
				\item В проекции на двумерную статическую плоскость, заданную векторами $(l_1, l_2)$ размерности n.
				\item В проекции на двумерную динамическую плоскость, заданную векторами $(l_1(t), l_2(t))$.
				\item Проекцию трубки достижимости на плоскость $(l_1, l_2)$ во времени $t \in [t_0, t_1]$.
				\item Проекцию трубки достижимости на плоскость $(l_1(t), l_2(t))$ во времени $t \in [t_0, t_1]$.
			\end{enumerate}
			
		Перебор направлений, в которых строится оценка, следует производить не по всему пространству, а в той плоскости, на которую строится проекция множества достижимости. Предусмотреть возможность изображения пересечения (объединения) некоторого количества оценок, взятых в различных направлениях, а также вывод отдельных оценок в одном выбранном направлении.
		Графический интерфейс программы можно не делать. Все входные данные и параметры должны задаваться в одном файле, а не во всей программе.
		В программе должны корректно обрабатываться ситуации, когда из-за фазовых ограничений трубки <<схлопываются>>.
		
        \textbf{Требования к отчету:}

        \begin{enumerate}
          \item В отчете все формулы, по которым строится оценка, должны быть выведены от начала и до конца.
          \item В качестве примера рассмотреть колебательную систему из варианта задания прошлого семестра. Параметры подобраться самостоятельно. Сравнить трубки достижимости при наличии фазового ограничения и без него. Проиллюстрировать разные случаи.
        \end{enumerate}
        
\section{Теоретическая часть}
	\section{Необходимые определения}
	\begin{definition}
		Множество $\{x \in  \mathbb{R}^n | \langle x - q, Q^{-1} (x - q) \rangle \leqslant 1 \}$, где $Q$ --- симметрическая, положительно определенная матрица, называется эллипсоидом с центром в точке $q \in  \mathbb{R}^n$ и матрицей $Q \in \mathbb{R}^{n \times n}$. Обозначение: $\mathcal{E} (q, Q)$.
	\end{definition}
	
	\begin{definition}
		Опорной функцией множества $M \subset \mathbb{R}^n$ называется $\rho (l | M ) = \sup\limits_{x \in M} \langle l, x \rangle$.
	\end{definition}	
	
	Опорной функцией эллипсоида $\mathcal{E} (q, Q)$ является $\rho (l | \mathcal{E} (q, Q)) = \langle l, q \rangle + \langle l, Ql \rangle^{\frac{1}{2}}$.
	
	\begin{definition}
		Полурасстоянием Хаусдорфа между двумя множествами $A$ и $B$ называется 
		$$
			h_{+} (A, B) = \inf \{ \varepsilon : A \subseteq B + B_{\varepsilon} (0) \}, 
		$$
		$$
			h_{-} (A, B) = h_{+} (B, A).
		$$
	\end{definition}
	
	\begin{definition}
		Расстоянием Хаусдорфа между двумя множествами $A$ и $B$ называют
		$$
			h (A, B) = \max \{ h_{+}, h_{-} \}.
		$$
	\end{definition}
	
	\begin{definition}
		Суммой по Минковскому множеств $A$ и $B$ называется такое множество $S$, что
		$$
			S = \cup_{a \in A, b \in B} \{a + b\}.
		$$
	\end{definition}
	
	\begin{definition}
		Геометрической разностью двух множеств $A \in \mathbb{R}^n$ и $B \in \mathbb{R}^n$ называется множество всех векторов $A - B = \{ x \in \mathbb{R}^n | \forall b \in B x + b \in A \}$. 
	\end{definition}
	Опорная функция геометрической разности множеств равна овыпукленной разности опорных функций исходных множеств. Овыпукленная функция --- максимальная функция среди всех выпуклых функций, меньших исходной:
	$$
		\rho (l | A - B) = conv (\rho (l | A) - \rho (l | B)).
	$$
			
		\subsection{Внутренняя оценка геометрической разности двух эллипсоидов}
		Рассмотрим геометрическую разность эллипсоидов $\mathcal{E}_1(q_1, Q_1) - \mathcal{E}_2(q_2, Q_2)$. Она может в общем случае не являться эллипсоидом. Однако, как и в случае разности, можно построить эллипсоид, явлюящийся внутренней аппроксимацией разности.
		Параметры эллипсоида внутренней оценки $\mathcal{E}_{-} (q, Q)$ определяются по формулам:
		$$
			q = q_1 - q_2, 
		$$
		$$
			Q = (p_1 - p_2) \left( \dfrac{1}{p_1} Q_1 - \dfrac{1}{p_2} Q_2 \right),
		$$
		где $p_1 p_2 > 0$. При этом, касание по направлению $l$ будет происходить при $p_i = \langle l, Q_i l \rangle^{\frac{1}{2}}$.
	
	\section{Метод решения задачи}
		\begin{definition}
			Множеством достижимости достижимости в момент времени $t$ называется множество $\mathcal{X} [t]$ всех точек $x$, в которые можно попасть из начального множества $\mathcal{X}_0$ в момент времени $t$ при выборе какого-либо допустимого управления $u$:
			$$
				\mathcal{X} = \left\{ x | \exists u(s) : t_0 \leqslant s \leqslant t \Rightarrow x(t, t_0, x_0) = x \right\}.
			$$
		\end{definition}	
		
		\begin{definition}
			Трубкой достижимости называется множество $X[\cdot] = \mathcal{X} [\cdot, t_0, \mathcal{X}_)]$.
		\end{definition}
		
		\begin{definition}
			Множеством достижимости при фазовых ограничениях в момент времени $t$ $Y(t)$ и начальном положении $(t_0, \mathcal{X}_0)$ называется множество
			$$
				\mathcal{X}[t] = \left\{ x | \exists u(s) : t_0 \leqslant s \leqslant t \Rightarrow x(t, t_0, x_0) = x \in Y(t) \right\}.
			$$
		\end{definition}
		Аналогично для трубки достижимости при фазовых ограничениях.
		
		Для решения задачи воспользуемся эволюционным уравнением:
		$$
			\lim_{\sigma \leftarrow 0} \dfrac{1}{\sigma} h \left\{ \mathcal{X}[t + \sigma] , \left(\mathcal{X}[t] + \sigma B(t) \mathcal{P}[t] \right) \cap \mathcal{Y}[t+\sigma] \right\} = 0.
		$$	 
	
		Предполагая непрерывность по Хаусдорфу множеств $\mathcal{X}[t]$ и $\mathcal{Y}[t]$, перепишем выражения для этих множеств для момента времени $t + \sigma$ в следующем виде:
		$$
			\mathcal{X}[t+\sigma] = \mathcal{X}[t] + \sigma A(t) \mathcal{X}[t] + \sigma B(t) \mathcal{P}[t], 
		$$
		Таким образом, исходное эволюционное уравнение эквивалентно следующему уравнению:
		$$
			\mathcal{X}[t+\sigma] = ((I + \sigma A(t)) \mathcal{X}[t] + \sigma B(t) \mathcal{P}[t]) \cap \mathcal{Y}[t+\sigma] + o(\sigma).
		$$

		Будем строить внутренние эллипсоидальные оценки множества достижимости. Пусть эллипсоид $\mathcal{E}_{-} (q_{-} [t], Q_{-} [t])$ --- внутренняя эллипсоиадальная аппроксимация множества достижимости в момент времени $t$ без фазовых ограничений. Тогда для момента времени $t + \sigma$ справедливо:
		$$
			\mathcal{E}_{-}(q_{-} [t + \sigma], Q_{-} [t + \sigma]) = ((I + \sigma A(t)) \mathcal{E}_{-} (q_{-} [t], Q_{-} [t]) + \sigma B(t) \mathcal{E}_{-} (p(t), P(t)) =
		$$ 		
		$$
			= \mathcal{E}_{-} ((I + \sigma A(t)) q_{-}[t], (I + \sigma A(t)) Q_{-}[t] (I + \sigma A(t))^{T}) + \mathcal{E}_{-} (\sigma B(t) p(t), \sigma B(t) P(t) \sigma B^{T}(t)) = 	
		$$
		$$
			= \mathcal{E}_{-} \left( q_1 + q_2, S_1 Q_1^{\frac{1}{2}} + S_2 Q_2^{\frac{1}{2}}\right),
		$$
		где 
		$$
			q_1 + q_2 = (I + \sigma A(t)) q_{-}[t] + \sigma B(t) p(t),
		$$
		$$
			Q_1 = I + \sigma A(t)) Q_{-}[t] (I + \sigma A(t))^{T},
		$$
		$$
			Q_2 = \sigma B(t) P(t) \sigma B^{T}(t),
		$$
		а матрицы $S_1$ и $S_2$ удовлетворяют следующим свойствам:
		$$
			S_i S_i^{T} = S_i^{T} S_i = I, i = 1,2.
		$$
		
		Данная формула дает возможность итерационного построения внутренней эллипсоидальной оценки множества достижимости --- с некоторым шагом $\sigma$ будем строить множество достижимости до тех пор, пока не достигнем момента времени $t_1$, а за начальное значение $\mathcal{X}[t]$ возьмем эллипсоид $\mathcal{E}(x_0, X_0)$.	
		
		Для того, чтобы выполнялись фазовые ограничения $\mathcal{Y}(t)$ на множество достижимости, будем на каждом шаге $t$ полученную оценку пересекать с множеством $\mathcal{Y}(t)$ и строить эллипсоидальную оценку пересечения двух множеств средствами Ellipsoidal Toolbox, а именно с помощью функции intersection$\_$ia. 
		
		Для того, чтобы касание эллипсоидальной оценки происходило по направлению $l, (l \in \mathbb{R}^n, || l || = 1)$, нужно чтобы выполнялось следующее соотношение:
		$$
			S_1 Q_1^{\frac{1}{2}} l = \lambda S_2 Q_2^{\frac{1}{2}} l, \lambda > 0.
		$$
		
		Поэтому, будем брать матрицу $S_1$ равную единичной, а матрицу $S_2$ находить из этого соотношения. Для этого воспользуемся функцией ell$\_$valign, входящей в состав Ellipsoidal Toolbox.
		
		Для более точного построения множества достижимости будем проводить перебор направлений, вдоль которых происходит касание эллипсоидальной оценки и множества. В силу того, что по условию задачи необходимо построить проекции трубки достижимости и множества достижимости на некоторую плоскость, порожденную векторами $l_1$ и $l_2$, то перебор направлений будем производить по единичной сфере, принадлежащей этой плоскости, а в соотношении для матриц $S_1$ и $S_2$ в качестве матриц $Q_1$ и $Q_2$ будем использовать проекции конфигурационных матриц эллипсоиадальных оценок, полученных из эволюционного уравнения. После этого полученные оценки будем объединять.
		
		Проекция матрицы $Q$ и вектора $q$ на плоскость $(l_1, l_2)$ вычисляются по следующим формулам:
		$$
			P = (l_1, l_2),
		$$
		$$
			\widehat{Q} = P' Q P,
		$$ 
		$$
			\widehat{q} = P' q.
		$$
							
        
        
\section{Вывод уравнений движений системы из примера}
        Построим уравнения движения маятника. Для этого возьмем за обобщенные координаты углы между вертикалью и положением первого и второго стержня и обозначим их за $\varphi_1$ и $\varphi_2$.
        Потенциальная энергия системы выражается как сумма потенциальных энергий первого и второго шариков:
        $$
            \Pi = - m g l_1 \cos \varphi_1 - m g (l_1 \cos \varphi_1 + l_2 \cos \varphi_2).
        $$
        Если выразить линейные скорости шариков через их угловые скорости, то получим, что
        $$
            v_1 = l_1 \dot{\varphi_1},
        $$
        $$
            v_2 = v_1 + l_2 \dot{\varphi_2} = l_1 \dot{\varphi_1} + l_2 \dot{\varphi_2}.
        $$
        Используя эти соотношения, построим выражение для кинетической энергии:
        $$
            K = \dfrac{m v_1^2}{2} + \dfrac{m v_2^2}{2} = \dfrac{m}{2} \left( 2 l_1^2 \dot{\varphi_1}^2 + 2 l_1 l_2 \dot{\varphi_1} \dot{\varphi_2} +  l_2^2 \dot{\varphi_2}^2\right).
        $$
        Выпишем лагранжиан системы и с помощью уравнений Лагранжа второго рода получим уравнения движения системы:
        $$
            L = K - \Pi = \dfrac{m}{2} \left( 2 l_1^2 \dot{\varphi_1}^2 + 2 l_1 l_2 \dot{\varphi_1} \dot{\varphi_2} +  l_2^2 \dot{\varphi_2}^2\right) + m g (2 l_1 \cos \varphi_1 + l_2 \cos \varphi_2).
        $$
        $$
            \dfrac{d}{dt} \left( \dfrac{\partial L}{\partial \dot{q_i}} - \dfrac{\partial L}{\partial q_i} \right) = 0,
        $$
        где за $q_i$ обозначены обобщенные координаты (в нашем случае --- $\varphi_1$ и $\varphi_2$).
        $$
            \dfrac{\partial L}{\partial \dot{\varphi_1}} = 2 m l_1^2 \dot{\varphi_1} + m l_1 l_2 \dot{\varphi_2},
        $$
        $$
            \dfrac{\partial L}{\partial \dot{\varphi_2}} = m l_2^2 \dot{\varphi_1} + m l_1 l_2 \dot{\varphi_1},
        $$
        $$
            \dfrac{\partial L}{\partial \varphi_1} = - 2 m g l_1 \sin \varphi_1,
        $$
        $$
            \dfrac{\partial L}{\partial \varphi_2} = - m g l_2 \sin \varphi_2.
        $$
        Окончательно получим:
        $$
            \begin{cases}
                2 m l_1^2 \ddot{\varphi_1} + m l_1 l_2 \ddot{\varphi_2} + 2 m g l_1 \sin \varphi_1 = 0, \\
                m l_2 \ddot{\varphi_2} + m l_1 l_2 \ddot{\varphi_1} + m g l_2 \sin \varphi_2 = 0.
            \end{cases}
        $$
        Сократим на $m$ оба уравнения и на $l_1$ и $l_2$ первое и второе уравнения соответственно, затем вычитая одно уравнение из другого получим два уравнения, в которые входят только $\ddot{\varphi_1}$ и $\ddot{\varphi}$:
        $$
            \begin{cases}
                l_1 \ddot{\varphi_1} + g (2 \sin \varphi_1 - \sin \varphi_2) = 0, \\
                l_2 \ddot{\varphi_2} - g (2 \sin \varphi_1 - 2 \sin \varphi_2) = 0.
            \end{cases}
        $$
        Приведем систему к нормальной форме:
        $$
            \begin{cases}
                \dot{\varphi_1} = \psi_1, \\
                \dot{\varphi_2} = \psi_2, \\
                \dot{\psi_1} = - g \dfrac{2\sin \varphi_1 - \sin \varphi_2}{l_1}, \\
                \dot{\psi_2} = g \dfrac{2\sin \varphi_1 - 2 \sin \varphi_2}{l_2}.
            \end{cases}
        $$
        В силу того, что по условию задачи маятник совершает малые колебания, можно заменить $sin \varphi$ на $\varphi$. Сделав это преобразование, получим окончательный вид системы, которая будет являться линейной и стационарной:
        $$
            \begin{cases}
                \dot{\varphi_1} = \psi_1, \\
                \dot{\varphi_2} = \psi_2, \\
                \dot{\psi_1} = - g \dfrac{2\varphi_1 - \varphi_2}{l_1}, \\
                \dot{\psi_2} = g \dfrac{2\varphi_1 - 2 \varphi_2}{l_2}.
            \end{cases}
        $$
    \subsection{Некоторые сведения из теории управления}
        Для начала запишем уравнения системы с использованием управления.
        Так как управляющее устройство прикреплено только ко второму шарику, то уравнения движения системы с управлением примут вид:
        $$
            \begin{cases}
                \dot{\varphi_1} = \psi_1, \\
                \dot{\varphi_2} = \psi_2, \\
                \dot{\psi_1} = - g \dfrac{2\varphi_1 - \varphi_2}{l_1}, \\
                \dot{\psi_2} = g \dfrac{2\varphi_1 - 2 \varphi_2}{l_2} + u.
            \end{cases}
        $$
\section{Примеры работы программы}
	\subsection{Пример 1}
		В данной системе матрицы $A$, $B$, $P$, $X_0$ являются единичными матрицам в $\mathbb{R}^3$, векторы $p$, $x_0$ --- нулевыми. Диапазон времени: $t_0 = 0$, $t_1 = 3$, фазовые ограничения отсутствуют. За статичные направления $l_1$ и $l_2$ взяты векторы $[1, 0, 0]$ и $[0, 1, 0]$, за динамичные --- $l_1(t) = [\sin(t); \cos(t); t], l_2(t) = [\cos(t); \sin(t); t]$.
	%\begin{center}
	%	\includegraphics[scale=0.75]{set_1.eps} \\
	%	Проекция множества достижимости на статическую плоскость $(l_1, l_2)$.
	%\end{center}		
	%\begin{center}
	%	\includegraphics[scale=0.75]{tube_1.eps} \\
	%	Проекция трубки достижимости на статическую плоскость $(l_1, l_2)$.
	\%end{center}			
	\subsection{Пример 2}
	В данной системе:
	$$A =
	\begin{pmatrix}
		1 & 0 & 0 \\
		0 & 2 & 0 \\
		0 & 0 & 1,
	\end{pmatrix}, 
	B = 
	\begin{pmatrix}
		1 & 0 & 0 \\
		0 & 1 & 0 \\
		0 & 0 & 1,
	\end{pmatrix}
	P = 
	\begin{pmatrix}
		1 & 0 & 0 \\
		0 & 1 & 0 \\
		0 & 0 & 1,
	\end{pmatrix}
	$$	
	$$
		x_0 = [0, 0, 0]^{T}, p = [0, 0, 0]^{T}, t_0 = 0, t_1 = 3.
	$$
	В данной системе есть фазовые ограничения $x_1 \leqslant 3$.
	%\begin{center}
	%	\includegraphics[scale=0.75]{set_2.eps} \\
	%	Проекция множества достижимости на статическую плоскость $(l_1, l_2)$.
	%\end{center}		
	%\begin{center}
	%	\includegraphics[scale=0.75]{tube_2_1.eps} \\
	%	Проекция трубки достижимости на статическую плоскость $(l_1, l_2)$.
	%\end{center}	
	%\begin{center}
	%	\includegraphics[scale=0.75]{tube_2_2.eps} \\
	%	Проекция трубки достижимости на статическую плоскость $(l_1, l_2)$.
	%\end{center}	
		
	\subsection{Пример 3}
		Рассмотрим колебательную систему из задания прошлого семестра. В этой системе:
	$$A =
	\begin{pmatrix}
		0 & 0 & 1 & 0 \\
		0 & 0 & 0 & 1 \\
		-2 \frac{g}{l_1} & \frac{g}{l_1} & 0 & 0 \\
		2 \frac{g}{l_2} & -2\frac{g}{l_2} & 0 & 0,
	\end{pmatrix}, 
	B = 
	\begin{pmatrix}
		0 \\
		0 \\
		0 \\ 
		1, 
	\end{pmatrix}
	P = 
	\begin{pmatrix}
		1 & 0 & 0 & 0 \\
		0 & 1 & 0 & 0\\
		0 & 0 & 1 & 0 \\
		0 & 0 & 0 & 1,
	\end{pmatrix}
	$$	
	$$
		x_0 = [0.1, 0.3, 1, 1]^{T}, p = [0, 0, 0, 0]^{T}, t_0 = 0, t_1 = 3, l_1 = 2, l_2 = 1.
	$$		
	На систему наложены фазовые ограничения $|x_1| \leqslant y_1$,  $y_1 = 1$.
	
	\begin{center}
%		\includegraphics[scale=0.75]{set_3.eps} \\
		Проекция множества достижимости на статическую плоскость $(l_1, l_2)$.
	\end{center}		
	\begin{center}
%		\includegraphics[scale=0.75]{tube_3_1.eps} \\
		Проекция трубки достижимости на статическую плоскость $(l_1, l_2)$.
	\end{center}	
	\begin{center}
%		\includegraphics[scale=0.75]{tube_3_2.eps} \\
		Проекция трубки достижимости на статическую плоскость $(l_1, l_2)$.
	\end{center}	
	
	Из рисунков трубки достижимости видно, что фазовые ограничения выполняются.
\section{Библиография}
  \begin{thebibliography}{99}
    \bibitem{newllang} \emph{Голубев~Ю.~Ф.} Основы теоретической механики: Учебник. 2-е изд., перераб. и дополн. --- М.:Изд-во МГУ, 2000.
    \bibitem{Modeling} \emph{P. Gagarinov, Alex A. Kurzhanskiy} Ellipsoial toolbox: ver. 2.0 beta 1, 2013.
  \end{thebibliography}
\end{document}
