\documentclass[a4paper, 14pt]{article}
\usepackage{a4wide}
\usepackage[utf8]{inputenc}
\usepackage[russian]{babel}
\usepackage[dvips]{graphicx, color}
\usepackage{epstopdf}
\usepackage{amsmath}
\usepackage{amsfonts}
\usepackage{indentfirst}
\usepackage{extsizes}
\usepackage{amssymb}
\usepackage{amsthm}
\usepackage{tikz}
\usepackage{pgfplots}
\usepackage{multirow} % улучшенное форматирование таблиц
\usepackage{ulem} % подчеркивания
\usepackage{geometry}
\geometry{left=3cm}
\geometry{right=1.5cm}
\geometry{top=2.4cm}
\geometry{bottom=2.4cm}

\begin{document}


\linespread{1.3} % полуторный интервал
\renewcommand{\rmdefault}{ftm} % Times New Roman
\frenchspacing
\thispagestyle{empty}

\begin{center}
\ \vspace{-3cm}

\includegraphics[width=0.5\textwidth]{msu.eps}\\

{\scshape Московский государственный университет имени М.~В.~Ломоносова}\\
Факультет вычислительной математики и кибернетики\\
Кафедра системного программирования

\vfill

{\LARGE Курсовая работа}

\vspace{1cm}

{\Huge\bfseries <<Определение языка сообщений социальной сети Twitter>>} \\

\end{center}

\vspace{1cm}

\begin{flushright}
  \large
  \textit{Выполнил студент 327 группы}\\
  М.~А.~Кольцов

  \vspace{5mm}

  \textit{Научный руководитель}\\
  В.~Д.~Майоров
\end{flushright}

\vfill

\begin{center}
Москва, 2014
\end{center}

\pagebreak
\tableofcontents
\pagebreak

\newtheorem{definition} {Определение}
\newtheorem{option} {Свойство}
\newtheorem{theorem} {Теорема}


\section{Введение}
	        В настоящее время человечество имеет доступ к огромному запасу знаний, накопленному за тысячи лет. Немалая часть этих знаний представляется
	        в виде текстов на различных языках. В связи с этим активно разрабатываются методы, предназначенные для автоматического извлечения
	        и преобразования информации, данной в символьном представлении. 
	        Возникло научное направление <<обработка естественных языков>>. Одной из его фундаментальных задач является определение языка текста. 


	        Стандартным подходом к этой задаче является применение машинного обучения. А именно, если у нас есть база из сотен документов 
	        на нескольких языках, то можно <<предсказать>> язык поступившего на рассмотрение документа, сравнив его с имеющимися. 
	        В общем случае, нужно по имеющимся данным построить так называемую модель, а затем все действия с текстами проводить в терминах 
	        этой модели. 
	        Классическим примером является метод, описанный в 1994 году: каждому документу сопоставим <<профиль документа>> --- упорядоченный по
	        числу встречаний список n-грамм --- последовательностей длины n подряд идущих символов. <<Профиль языка>> --- это совокупность профилей
	        документов, которые имеются на этом языке. Теперь, если нужно для какого-то документа определить язык, то последовательность
	        действий такова:
	        \begin{enumerate}
	        		\item Составляется профиль этого документа
	        		\item Сравнивается с имеющимися профилями языков
	        		\item Тот язык, чей профиль наиболее похож на профиль документа, объявляется результатом
	        \end{enumerate}
	          
	        Вышеописанный метод плохо работает для коротких сообщений. В то же время, поток информации в виде коротких шумных сообщений нельзя игнорировать ---
	        в социальной сети Twitter среднее количество сообщений в день составляет примерно 58 000 000, а длина каждого ограничена 140 символами. 
	        Такой формат текстов обусловил появление новых алгоритмов. В данной работе рассматриваются современные методы 
	        решения задачи определения языка, а также предлагается улучшение для одного из них. Проводится тестирование, показывающее
	        превосходство по полноте распознавания языка полученного результата над существующими.
  
	        
\section{Постановка задачи}
		\subsection{Формальное описание задачи автоматического определения языка}
		Пусть $L$ - множество меток, сопоставленных естественным языкам.
		По заданному тренировочному корпусу $$ T = \{(msg_{1}, lang_{1}), (msg_{2}, lang_{2}), \ldots, (msg_{N}, lang_{N}) \} $$
		(здесь $msg_{i}, i \in \overline{1..N},$ - текст на естественном языке, $lang_{i} \in L$ - метка этого языка) нужно построить классификатор,
		который произвольному входному сообщению $new\_msg$ на языке $some\_lang$ сопоставит метку $l \in L$, соответствующую этому языку,
		 или же сообщит, что язык текста невозможно достоверно распознать.
		\subsection{Цели и задачи курсовой работы}
		В данной работе в качестве текстов выступают так называемые <<твиты>> - сообщения из социальной сети Twitter\footnote{https://twitter.com/}, 
		а множество $L$ соответствует 18
		языкам, которые можно разделить на три группы по типу алфавита:
		\begin{itemize}
			\item Кириллические: болгарский, чувашский, русский, татарский, украинский
			\item Арабские: арабский, персидский (фарси), хинди, маратхи, непальский, урду
			\item Латинские: нидерландский, французский, английский, немецкий, итальянский, испанский, турецкий
		\end{itemize}
		Цели работы:
		\begin{enumerate}
			\item Исследовать современные решения задачи автоматического определения языка коротких сообщений
			\item Провести совместное сравнительное тестирование некоторых методов решения задачи и выяснить, действительно ли они показывают
			заявленное авторами качество классификации
			\item Исследовать возможность улучшения какого-либо алгоритма решения задачи автоматического определения языка коротких сообщений	
		\end{enumerate}

\section{Метод решения задачи}
		\begin{definition}
			Множеством достижимости достижимости в момент времени $t$ называется множество $\mathcal{X} [t]$ всех точек $x$, в которые можно попасть из начального множества $\mathcal{X}_0$ в момент времени $t$ при выборе какого-либо допустимого управления $u$:
			$$
				\mathcal{X} = \left\{ x | \exists u(s) : t_0 \leqslant s \leqslant t \Rightarrow x(t, t_0, x_0) = x \right\}.
			$$
		\end{definition}	
		
		\begin{definition}
			Трубкой достижимости называется множество $X[\cdot] = \mathcal{X} [\cdot, t_0, \mathcal{X}_)]$.
		\end{definition}
		
		\begin{definition}
			Множеством достижимости при фазовых ограничениях в момент времени $t$ $Y(t)$ и начальном положении $(t_0, \mathcal{X}_0)$ называется множество
			$$
				\mathcal{X}[t] = \left\{ x | \exists u(s) : t_0 \leqslant s \leqslant t \Rightarrow x(t, t_0, x_0) = x \in Y(t) \right\}.
			$$
		\end{definition}
		Аналогично для трубки достижимости при фазовых ограничениях.
		
		Для решения задачи воспользуемся эволюционным уравнением:
		$$
			\lim_{\sigma \leftarrow 0} \dfrac{1}{\sigma} h \left\{ \mathcal{X}[t + \sigma] , \left(\mathcal{X}[t] + \sigma B(t) \mathcal{P}[t] \right) \cap \mathcal{Y}[t+\sigma] \right\} = 0.
		$$	 
	
		Предполагая непрерывность по Хаусдорфу множеств $\mathcal{X}[t]$ и $\mathcal{Y}[t]$, перепишем выражения для этих множеств для момента времени $t + \sigma$ в следующем виде:
		$$
			\mathcal{X}[t+\sigma] = \mathcal{X}[t] + \sigma A(t) \mathcal{X}[t] + \sigma B(t) \mathcal{P}[t], 
		$$
		Таким образом, исходное эволюционное уравнение эквивалентно следующему уравнению:
		$$
			\mathcal{X}[t+\sigma] = ((I + \sigma A(t)) \mathcal{X}[t] + \sigma B(t) \mathcal{P}[t]) \cap \mathcal{Y}[t+\sigma] + o(\sigma).
		$$

		Будем строить внутренние эллипсоидальные оценки множества достижимости. Пусть эллипсоид $\mathcal{E}_{-} (q_{-} [t], Q_{-} [t])$ --- внутренняя эллипсоиадальная аппроксимация множества достижимости в момент времени $t$ без фазовых ограничений. Тогда для момента времени $t + \sigma$ справедливо:
		$$
			\mathcal{E}_{-}(q_{-} [t + \sigma], Q_{-} [t + \sigma]) = ((I + \sigma A(t)) \mathcal{E}_{-} (q_{-} [t], Q_{-} [t]) + \sigma B(t) \mathcal{E}_{-} (p(t), P(t)) =
		$$ 		
		$$
			= \mathcal{E}_{-} ((I + \sigma A(t)) q_{-}[t], (I + \sigma A(t)) Q_{-}[t] (I + \sigma A(t))^{T}) + \mathcal{E}_{-} (\sigma B(t) p(t), \sigma B(t) P(t) \sigma B^{T}(t)) = 	
		$$
		$$
			= \mathcal{E}_{-} \left( q_1 + q_2, S_1 Q_1^{\frac{1}{2}} + S_2 Q_2^{\frac{1}{2}}\right),
		$$
		где 
		$$
			q_1 + q_2 = (I + \sigma A(t)) q_{-}[t] + \sigma B(t) p(t),
		$$
		$$
			Q_1 = I + \sigma A(t)) Q_{-}[t] (I + \sigma A(t))^{T},
		$$
		$$
			Q_2 = \sigma B(t) P(t) \sigma B^{T}(t),
		$$
		а матрицы $S_1$ и $S_2$ удовлетворяют следующим свойствам:
		$$
			S_i S_i^{T} = S_i^{T} S_i = I, i = 1,2.
		$$
		
		Данная формула дает возможность итерационного построения внутренней эллипсоидальной оценки множества достижимости --- с некоторым шагом $\sigma$ будем строить множество достижимости до тех пор, пока не достигнем момента времени $t_1$, а за начальное значение $\mathcal{X}[t]$ возьмем эллипсоид $\mathcal{E}(x_0, X_0)$.	
		
		Для того, чтобы выполнялись фазовые ограничения $\mathcal{Y}(t)$ на множество достижимости, будем на каждом шаге $t$ полученную оценку пересекать с множеством $\mathcal{Y}(t)$ и строить эллипсоидальную оценку пересечения двух множеств средствами Ellipsoidal Toolbox, а именно с помощью функции intersection$\_$ia. 
		
		Для того, чтобы касание эллипсоидальной оценки происходило по направлению $l, (l \in \mathbb{R}^n, || l || = 1)$, нужно чтобы выполнялось следующее соотношение:
		$$
			S_1 Q_1^{\frac{1}{2}} l = \lambda S_2 Q_2^{\frac{1}{2}} l, \lambda > 0.
		$$
		
		Поэтому, будем брать матрицу $S_1$ равную единичной, а матрицу $S_2$ находить из этого соотношения. Для этого воспользуемся функцией ell$\_$valign, входящей в состав Ellipsoidal Toolbox.
		
		Для более точного построения множества достижимости будем проводить перебор направлений, вдоль которых происходит касание эллипсоидальной оценки и множества. В силу того, что по условию задачи необходимо построить проекции трубки достижимости и множества достижимости на некоторую плоскость, порожденную векторами $l_1$ и $l_2$, то перебор направлений будем производить по единичной сфере, принадлежащей этой плоскости, а в соотношении для матриц $S_1$ и $S_2$ в качестве матриц $Q_1$ и $Q_2$ будем использовать проекции конфигурационных матриц эллипсоиадальных оценок, полученных из эволюционного уравнения. После этого полученные оценки будем объединять.
		
		Проекция матрицы $Q$ и вектора $q$ на плоскость $(l_1, l_2)$ вычисляются по следующим формулам:
		$$
			P = (l_1, l_2),
		$$
		$$
			\widehat{Q} = P' Q P,
		$$ 
		$$
			\widehat{q} = P' q.
		$$
							
        
        
\section{Вывод уравнений движений системы из примера}
        Построим уравнения движения маятника. Для этого возьмем за обобщенные координаты углы между вертикалью и положением первого и второго стержня и обозначим их за $\varphi_1$ и $\varphi_2$.
        Потенциальная энергия системы выражается как сумма потенциальных энергий первого и второго шариков:
        $$
            \Pi = - m g l_1 \cos \varphi_1 - m g (l_1 \cos \varphi_1 + l_2 \cos \varphi_2).
        $$
        Если выразить линейные скорости шариков через их угловые скорости, то получим, что
        $$
            v_1 = l_1 \dot{\varphi_1},
        $$
        $$
            v_2 = v_1 + l_2 \dot{\varphi_2} = l_1 \dot{\varphi_1} + l_2 \dot{\varphi_2}.
        $$
        Используя эти соотношения, построим выражение для кинетической энергии:
        $$
            K = \dfrac{m v_1^2}{2} + \dfrac{m v_2^2}{2} = \dfrac{m}{2} \left( 2 l_1^2 \dot{\varphi_1}^2 + 2 l_1 l_2 \dot{\varphi_1} \dot{\varphi_2} +  l_2^2 \dot{\varphi_2}^2\right).
        $$
        Выпишем лагранжиан системы и с помощью уравнений Лагранжа второго рода получим уравнения движения системы:
        $$
            L = K - \Pi = \dfrac{m}{2} \left( 2 l_1^2 \dot{\varphi_1}^2 + 2 l_1 l_2 \dot{\varphi_1} \dot{\varphi_2} +  l_2^2 \dot{\varphi_2}^2\right) + m g (2 l_1 \cos \varphi_1 + l_2 \cos \varphi_2).
        $$
        $$
            \dfrac{d}{dt} \left( \dfrac{\partial L}{\partial \dot{q_i}} - \dfrac{\partial L}{\partial q_i} \right) = 0,
        $$
        где за $q_i$ обозначены обобщенные координаты (в нашем случае --- $\varphi_1$ и $\varphi_2$).
        $$
            \dfrac{\partial L}{\partial \dot{\varphi_1}} = 2 m l_1^2 \dot{\varphi_1} + m l_1 l_2 \dot{\varphi_2},
        $$
        $$
            \dfrac{\partial L}{\partial \dot{\varphi_2}} = m l_2^2 \dot{\varphi_1} + m l_1 l_2 \dot{\varphi_1},
        $$
        $$
            \dfrac{\partial L}{\partial \varphi_1} = - 2 m g l_1 \sin \varphi_1,
        $$
        $$
            \dfrac{\partial L}{\partial \varphi_2} = - m g l_2 \sin \varphi_2.
        $$
        Окончательно получим:
        $$
            \begin{cases}
                2 m l_1^2 \ddot{\varphi_1} + m l_1 l_2 \ddot{\varphi_2} + 2 m g l_1 \sin \varphi_1 = 0, \\
                m l_2 \ddot{\varphi_2} + m l_1 l_2 \ddot{\varphi_1} + m g l_2 \sin \varphi_2 = 0.
            \end{cases}
        $$
        Сократим на $m$ оба уравнения и на $l_1$ и $l_2$ первое и второе уравнения соответственно, затем вычитая одно уравнение из другого получим два уравнения, в которые входят только $\ddot{\varphi_1}$ и $\ddot{\varphi}$:
        $$
            \begin{cases}
                l_1 \ddot{\varphi_1} + g (2 \sin \varphi_1 - \sin \varphi_2) = 0, \\
                l_2 \ddot{\varphi_2} - g (2 \sin \varphi_1 - 2 \sin \varphi_2) = 0.
            \end{cases}
        $$
        Приведем систему к нормальной форме:
        $$
            \begin{cases}
                \dot{\varphi_1} = \psi_1, \\
                \dot{\varphi_2} = \psi_2, \\
                \dot{\psi_1} = - g \dfrac{2\sin \varphi_1 - \sin \varphi_2}{l_1}, \\
                \dot{\psi_2} = g \dfrac{2\sin \varphi_1 - 2 \sin \varphi_2}{l_2}.
            \end{cases}
        $$
        В силу того, что по условию задачи маятник совершает малые колебания, можно заменить $sin \varphi$ на $\varphi$. Сделав это преобразование, получим окончательный вид системы, которая будет являться линейной и стационарной:
        $$
            \begin{cases}
                \dot{\varphi_1} = \psi_1, \\
                \dot{\varphi_2} = \psi_2, \\
                \dot{\psi_1} = - g \dfrac{2\varphi_1 - \varphi_2}{l_1}, \\
                \dot{\psi_2} = g \dfrac{2\varphi_1 - 2 \varphi_2}{l_2}.
            \end{cases}
        $$
    \subsection{Некоторые сведения из теории управления}
        Для начала запишем уравнения системы с использованием управления.
        Так как управляющее устройство прикреплено только ко второму шарику, то уравнения движения системы с управлением примут вид:
        $$
            \begin{cases}
                \dot{\varphi_1} = \psi_1, \\
                \dot{\varphi_2} = \psi_2, \\
                \dot{\psi_1} = - g \dfrac{2\varphi_1 - \varphi_2}{l_1}, \\
                \dot{\psi_2} = g \dfrac{2\varphi_1 - 2 \varphi_2}{l_2} + u.
            \end{cases}
        $$
\section{Примеры работы программы}
	\subsection{Пример 1}
		В данной системе матрицы $A$, $B$, $P$, $X_0$ являются единичными матрицам в $\mathbb{R}^3$, векторы $p$, $x_0$ --- нулевыми. Диапазон времени: $t_0 = 0$, $t_1 = 3$, фазовые ограничения отсутствуют. За статичные направления $l_1$ и $l_2$ взяты векторы $[1, 0, 0]$ и $[0, 1, 0]$, за динамичные --- $l_1(t) = [\sin(t); \cos(t); t], l_2(t) = [\cos(t); \sin(t); t]$.
	%\begin{center}
	%	\includegraphics[scale=0.75]{set_1.eps} \\
	%	Проекция множества достижимости на статическую плоскость $(l_1, l_2)$.
	%\end{center}		
	%\begin{center}
	%	\includegraphics[scale=0.75]{tube_1.eps} \\
	%	Проекция трубки достижимости на статическую плоскость $(l_1, l_2)$.
	\%end{center}			
	\subsection{Пример 2}
	В данной системе:
	$$A =
	\begin{pmatrix}
		1 & 0 & 0 \\
		0 & 2 & 0 \\
		0 & 0 & 1,
	\end{pmatrix}, 
	B = 
	\begin{pmatrix}
		1 & 0 & 0 \\
		0 & 1 & 0 \\
		0 & 0 & 1,
	\end{pmatrix}
	P = 
	\begin{pmatrix}
		1 & 0 & 0 \\
		0 & 1 & 0 \\
		0 & 0 & 1,
	\end{pmatrix}
	$$	
	$$
		x_0 = [0, 0, 0]^{T}, p = [0, 0, 0]^{T}, t_0 = 0, t_1 = 3.
	$$
	В данной системе есть фазовые ограничения $x_1 \leqslant 3$.
	%\begin{center}
	%	\includegraphics[scale=0.75]{set_2.eps} \\
	%	Проекция множества достижимости на статическую плоскость $(l_1, l_2)$.
	%\end{center}		
	%\begin{center}
	%	\includegraphics[scale=0.75]{tube_2_1.eps} \\
	%	Проекция трубки достижимости на статическую плоскость $(l_1, l_2)$.
	%\end{center}	
	%\begin{center}
	%	\includegraphics[scale=0.75]{tube_2_2.eps} \\
	%	Проекция трубки достижимости на статическую плоскость $(l_1, l_2)$.
	%\end{center}	
		
	\subsection{Пример 3}
		Рассмотрим колебательную систему из задания прошлого семестра. В этой системе:
	$$A =
	\begin{pmatrix}
		0 & 0 & 1 & 0 \\
		0 & 0 & 0 & 1 \\
		-2 \frac{g}{l_1} & \frac{g}{l_1} & 0 & 0 \\
		2 \frac{g}{l_2} & -2\frac{g}{l_2} & 0 & 0,
	\end{pmatrix}, 
	B = 
	\begin{pmatrix}
		0 \\
		0 \\
		0 \\ 
		1, 
	\end{pmatrix}
	P = 
	\begin{pmatrix}
		1 & 0 & 0 & 0 \\
		0 & 1 & 0 & 0\\
		0 & 0 & 1 & 0 \\
		0 & 0 & 0 & 1,
	\end{pmatrix}
	$$	
	$$
		x_0 = [0.1, 0.3, 1, 1]^{T}, p = [0, 0, 0, 0]^{T}, t_0 = 0, t_1 = 3, l_1 = 2, l_2 = 1.
	$$		
	На систему наложены фазовые ограничения $|x_1| \leqslant y_1$,  $y_1 = 1$.
	
	\begin{center}
%		\includegraphics[scale=0.75]{set_3.eps} \\
		\begin{tikzpicture}
		\begin{axis}[xlabel=errors, ylabel=percentage]
		\addplot[blue!80!black] coordinates
			{(-23, 1.5) (0, 1.7) (10, 1.96)};
		\addlegendentry{liga}
		\end{axis}
		\end{tikzpicture}
		Проекция множества достижимости на статическую плоскость $(l_1, l_2)$.
	\end{center}		
	\begin{center}
%		\includegraphics[scale=0.75]{tube_3_1.eps} \\
		Проекция трубки достижимости на статическую плоскость $(l_1, l_2)$.
	\end{center}	
	\begin{center}
%		\includegraphics[scale=0.75]{tube_3_2.eps} \\
		Проекция трубки достижимости на статическую плоскость $(l_1, l_2)$.
	\end{center}	
	
	Из рисунков трубки достижимости видно, что фазовые ограничения выполняются.
\section{Библиография}
  \begin{thebibliography}{99}
    \bibitem{newllang} \emph{Голубев~Ю.~Ф.} Основы теоретической механики: Учебник. 2-е изд., перераб. и дополн. --- М.:Изд-во МГУ, 2000.
    \bibitem{Modeling} \emph{P. Gagarinov, Alex A. Kurzhanskiy} Ellipsoial toolbox: ver. 2.0 beta 1, 2013.
  \end{thebibliography}
\end{document}
