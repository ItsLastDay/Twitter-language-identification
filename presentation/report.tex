\documentclass[mathserif,utf8,14pt]{beamer}
\usepackage[utf8]{inputenc}
\usepackage{tikz}
\usepackage[english,russian]{babel}
\usepackage[absolute,overlay]{textpos}
\usepackage{amsmath}

\usetheme{Madrid}
%\setbeamerfont{framesource}{size=\small}
%\setbeamercolor{framesource}{fg=gray}
%\newcommand{\source}[1]{\begin{textblock*}{8cm}(0.5cm,8.7cm)
%    \begin{beamercolorbox}[ht=0.5cm,left]{framesource}
%        \usebeamerfont{framesource}\usebeamercolor[fg]{framesource} source: \itshape{{#1}}
%    \end{beamercolorbox}
%\end{textblock*}}

\makeatletter
\setbeamertemplate{footline}
{
  \leavevmode%
  \hspace{0.92\paperwidth}
  \hbox{%
%  \begin{beamercolorbox}[wd=.333333\paperwidth,ht=2.25ex,dp=1ex,center]{author in head/foot}%
%    \usebeamerfont{author in head/foot}\insertshortauthor%~~\beamer@ifempty{\insertshortinstitute}{}{(\insertshortinstitute)}
%  \end{beamercolorbox}%
%  \begin{beamercolorbox}[wd=.333333\paperwidth,ht=2.25ex,dp=1ex,center]{title in head/foot}%
%    \usebeamerfont{title in head/foot}\insertshorttitle
%  \end{beamercolorbox}%
  \begin{beamercolorbox}[wd=.08\paperwidth,ht=2.25ex,dp=1ex,right]{date in head/foot}%
%    \usebeamerfont{date in head/foot}\insertshortdate{}\hspace*{2em}
    \insertframenumber{} / \inserttotalframenumber\hspace*{2ex} 
  \end{beamercolorbox}}%
  \vskip0pt%
}
\makeatother

\beamertemplatenavigationsymbolsempty

\title{Определение языка сообщений социальной сети Twitter}
\author{Выполнил студент 327 группы \\ Михаил Кольцов}
\institute{
Научный руководитель: \\
Владимир Майоров
}
\vspace*{10px}
\date{\today}
\begin{document}
\begin{frame}
    \maketitle
\end{frame}


\begin{frame}{Введение}
    Twitter~--- социальная сеть, в которой более 200 миллионов пользователей 
    обмениваются сообщениями, которые называются <<твитами>>: \\
    \begin{block}{Пример твита}
        {\color{red}RT @TheWritingDiva:} {\color{blue}\#BlogPaws \#blogpawty} fiona waits 
        for treats {\color{magenta}@susandaffron} {\color{violet}pic.twitter.com/p41wMgEdq8}
    \end{block}
    {\color{red}Ретвит}, {\color{blue}хэштег}, {\color{violet}ссылка}, {\color{magenta}упоминание}.
\end{frame}

\begin{frame}{Введение. Почему Twitter?}
    \begin{itemize}
        \item Отслеживание общественного мнения (новый фильм, реформа,~$\ldots$)
        \item Обнаружение событий (землетрясения, пробки,~$\ldots$)
        \item Выявление социальных слоёв (либералов, вегетарианцев,~$\ldots$)
        \item И многое другое 
    \end{itemize}
\end{frame}

\begin{frame}{Введение. Трудности}
    \begin{itemize}
        \item Длина сообщения~--- до 140 символов
        \item Пользователи допускают ошибки
        \item Произвольное проявление эмоций
    \end{itemize}
    Необходимая мера~--- \emph{нормализация}:
    \begin{block}{Пример нормализации}
        {\color{red}"The Comeback"\ is coming back to HBO and DREAMS DO COME TRUE AFTER ALL!!!!!!!!} $\to$ 
        the comeback is 
        coming back to hbo and dreams do come true after all
    \end{block}
\end{frame}

\begin{frame}{Постановка задачи}
    Цель работы~--- исследование и сравнение методов автоматического определения языка сообщений Twitter
    \begin{enumerate}
        \item Исследовать существующие методы определения языка текста
        \item Провести совместное сравнительное тестирование некоторых методов
        \item Исследовать зависимость качества классификации от метода нормализации и кол-ва твитов в
            обучающей выборке
        \item Исследовать возможность улучшения какого-либо метода
    \end{enumerate}
\end{frame}

\begin{frame}{Сравниваемые системы}
    \begin{enumerate}
        \item \emph{TextCat}~--- создан в 1997 г., основан на подсчёте частоты 1..4-грамм
        \item \emph{Google CLD2}~--- создан в 2011 г., встроен в Google Chrome и Google Translate; основан на Naive Bayes
        \item \emph{Langid}~--- создан в 2011 г.; основан на Naive Bayes
        \item \emph{LIGA}~--- реализованный в рамках курсовой работы подход, основанный на графах, предложен в 2011~г.
        \item \emph{LIGAv2}~--- предлагаемое улучшение метода LIGA
        \item \emph{LogR}~--- реализованный в рамках курсовой работы метод, использующий логистическую регрессию, предложен в 2012 г.
    \end{enumerate}
\end{frame}

\begin{frame}{Корпус. Источники}
    \begin{enumerate}
        \item Научные статьи
        \item Twitter API
        \item Indigenous Tweets
    \end{enumerate}
\end{frame}

\begin{frame}{Корпус. Статистика}
    18 языков, 4 различных стиля написания: 
    \begin{itemize}
        \item Основанные на кириллице: болгарский, чувашский, русский, татарский, украинский (всего~205 175 твитов)
        \item Арабские: арабский, персидский (фарси), урду (всего~4605 твитов)
        \item Латинские: нидерландский, французский, английский, немецкий, итальянский, испанский, турецкий 
            (всего~13 382 твита)
        \item Деванагари: хинди, маратхи, непальский (всего~4041 твит)
    \end{itemize}
\end{frame}

\begin{frame}{Система тестирования}
    Написан набор скриптов на Bash и Python, обеспечивающих лёгкое добавление/удаление тестируемой системы или
    изменение параметров нормализации, предназначенный для сбора характеристик сравниваемых систем. 

    Два результирующих представления для твитов: 
    \begin{itemize}
        \item plain text
        \item plain text + извлечённая метаинформация
    (имя пользователя, местоположение, $\ldots$)
    \end{itemize}
    \begin{block}{Схема работы с твитами}
        plain\_text\_getters $\to$ merge\_files $\to$ normalize\_text $\to$ gen\_stat $\to$ main 
        $\to$ {\color{blue}результат}
    \end{block}
\end{frame}

\begin{frame}{Тестирование. Результаты}

    \begin{center}
        \small
    \begin{table}[h]
    \begin{tabular*}{1.029\textwidth}{| l| *{6}{|c} |}
    \hline 
    Мера & CLD2 & langid.py & LIGAv2 & LIGA & TextCat  & LogR\\
    \hline
    \emph{Точность} & 85.2\% & 78.4\% & 94.3\% & 90.8\% & 93.8\% & 29.5\%\\
    \emph{Полнота} & 79.4\% & 78.7\% & 94.0\% & 89.3\% & 90.5\% & 29.9\%\\
    \emph{F1-мера} & 82.2\% & 78.6\% & {\color{blue}94.2\%} & 90.1\% & 92.1\% & {\color{red}29.7\%}\\
    \emph{Accuracy} & 79.5\% & 78.5\% & 94.1\% & 89.6\% & 90.5\% & 29.4\%\\
    \hline
    \end{tabular*}
    \end{table}
    \end{center}
\end{frame}

\begin{frame}{Тестирование. Примеры ошибок}
    
            \begin{enumerate}
                \item \textbf{Похожесть языков}: \textit{lang leve ikea} (nl $\to$ en), 
                            \textit{state department condemns concerted campaign intimidate international journalists cairo} (en $\to$ fr).
                    \item \textbf{Наличие имён собственных на другом языке}:
                            \textit{moyes ponders toffees selection everton boss david moyes will have decisions make both ends the pitch} (en $\to$ fr).
                        \item \textbf{Фразы на другом языке}:
                            \textit{харесах видеоклип watch mario balotelli failed backheel} (bg $\to$ en).
                        \item \textbf{Неверная разметка}: 
                            \textit{встиг підстригтися уже наступний раз записався} (ru $\to$ uk). 
                        \item \textbf{Малая длина}:
                            \textit{кис}, \textit{прекратите}, \textit{lesles}, \textit{haa dus}, \textit{lool grave}. 
            \end{enumerate}
\end{frame}

\begin{frame}{Выводы}
    \begin{enumerate}
        \item Исследованы методы автоматического определения языка сообещний Twitter и произведено сравненительное тестирование некоторых из них
        \item Реализовано два метода в рамах курсовой работы
        \item Предложено улучшение для одного из методов, которое показывает качество классификации выше всех остальных
        \item Реализована система для тестирования методов классификации твитов
        \item Чтобы показать трудность решения задачи определения языка применительно к Twitter, проведён анализ ошибок классификации
    \end{enumerate}
\end{frame}


\begin{frame}{}
    \begin{center}
    \large Спасибо за внимание!
    \end{center}
\end{frame}














\end{document}
